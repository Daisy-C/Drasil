\chapter{A look under the hood - Our process}

The first step in removing unnecessary redundancy is identifying exactly what
that redundancy is, and where it exists. To that end we need to understand what
each of our software artifacts is attempting to communicate, who their audience
is, and what information can be considered boilerplate versus system-specific.
  **NOTE: in the breakdown make sure to mention "We actually found that some
  info is boilerplate, some is system-specific, and some is general to all
  members of a software family, but more specific than generic boilerplate"
  **NOTE: Also ensure each artifact has a 'who' (audience), 'what' (problem
  being solved), and 'how' (specific-knowledge vs boilerplate)
Luckily, we have an excellent starting point thanks to the work of many smart
people - artifact templates.

Lots of work has been done to specify exactly what should be documented in a
given artifact in an effort for standardization. Ironically, this has led to
many different 'standardized' templates. However, through the examination of a
number of templates for different artifact types, we have concluded they convey
roughly the same overall information for a given artifact. Most differences are
stylistic or related to content organization, as we will demonstrate in the 
following sections.
  --Or naming conventions

Once we understand our artifacts, we take a practical, example-driven approach
to identifying redundancy through the use of existing software system case
studies. For each of these case studies, we start by examining the source code
and existing software artifacts to understand exactly what problem they are
trying to solve. From there, we attempt to distill the system-specific knowledge
and generalize the boilerplate.

\section{A (very) brief introduction to our case study systems}
  
  [ Potentially canibalize the intro from the next section to specify the
    templates in use, so it doesn't seem to come out of nowhere.
    
    Summarize the following:
      -- Case study name
      -- Problem being solved
      -- Appendices containing artifacts? -- May come in handy for examples
  ]
  
To simplify the process of identifying redundancies and patterns, we have chosen
case studies developed using common artifact templates, specifically those 
used by [SmithEtAl] [source]. Also, as mentioned in [SCOPE], we have chosen
software systems that follow the 'input' -> 'process' -> 'output' pattern.
These systems cover a variety of use cases, to help avoid over-specializing into
one particular system-type. 

This section is meant to be used as a high-level reference to each case study, 
providing the general details at a glance. For the specifics of each system, 
all relevant case study artifacts can be found in the appendices.

** Should this be a table / series of cards?
\card{GlassBR}
{We need to efficiently and correctly predict whether a glass 
slab can withstand a blast under given conditions.}
{\ds{TODO}}

\card{SWHS}
{Solar water heating systems incorporating phase change 
 material (PCM) use a renewable energy source and provide a novel way of 
 storing energy. A system is needed to investigate the effect of employing PCM
 within a solar water heating tank.}
{\ds{TODO}}

\card{NoPCM}
{Solar water heating systems provide a novel way of 
heating water and storing renewable energy. A system is needed to investigate
the heating of water within a solar water heating tank.}
{\ds{TODO}}

\card{SSP}
{A slope of geological mass, composed of soil and rock 
 and sometimes water, is subject to the influence of gravity on the mass. 
 This can cause instability in the form of soil or rock movement which can
 be hazardous. A system is needed to evaluate the factor of safety of 
 a slope's slip surface and identify the critical slip surface of the slope, 
 as well as the interslice normal force and shear force along the critical 
 slip surface.}
{\ds{TODO}}

\card{Projectile}
{A system is needed to efficiently and correctly predict
 the landing position of a projectile.}
{\ds{TODO}}

\card{GamePhys}
{Many video games need physics libraries that simulate 
 objects acting under various physical conditions, while simultaneously being 
 fast and efficient enough to work in soft real-time during the game. 
 Developing a physics library from scratch takes a long period of time and is 
 very costly, presenting barriers of entry which make it difficult for game 
 developers to include physics in their products.}
{\ds{TODO}}

The majority of the aforementioned case studies were developed to solve real
problems, though there are a couple of exceptions.
  **NOTE: trying to find a good way to say 'these are not just things we cooked
    up to make Drasil look good'
    
The NoPCM case study was created as a software family member for the SWHS case
study. It was manually written, removing all references to PCM and thus 
remodeling the system.

The Projectile case study, however, was the first example of a system created
solely in Drasil (there was no manually created version to compare and 
contrast). As such, it will not be referenced often until DRASILSECTION since
it did not inform Drasil's design or development until much further in the 
process. The Projectile case study was created so we'd have a simple, 
understandable example for a general audience (it requires, at most, a 
high-school level understanding of physics). 

With our carefully selected case studies in hand we were able to begin our 
practical approach to finding and removing redundancies.

\section{Breaking down artifacts}
As noted earlier, for our approach to work we must understand exactly what each
of our artifacts are trying to say and to whom. By selecting our case studies 
from those developed using common artifact templates, we have given ourselves a
head start on that process, however, there is still much work to be done.

To start, we look at the Software Requirements Specification (SRS). The SRS
(or some incarnation of it) is one of the most important artifacts for any
software project as it specifies what problem the software is trying to solve.
There are many ways to state this problem, and [SmithEtAl] have given us a
strong recommendation of what to use as a starting point in their template.
[Figure] shows the table of contents for an SRS using the [SmithEtAl] template.

[Figure showing the ToC of SmithEtAl template]

With the structure of the document in mind, let us look at several of our case
studies' SRS documents to get a deeper understanding of what each section truly
represents. [Figure] shows the first section of [one/several of our case studies, rest in Appendices].
[ The case studies themselves will be introduced in more detail in later sections, but
keep in mind at this point, we don't care about the superficial differences. -- Not exactly true if we introduce them in previous section ]
 We are strictly looking for patterns! Patterns will give us insight
into the root of *what* is being said in each section.

[Figure showing the Ref Section of at least 3 (preferably 4-6) case studies, may
need to be split into multiple, or have them somehow cropped/overlayed/arranged
to be not terrible to look at - TBD]

Looking at the Table of Symbols, Table of Units, and Table of Abbreviations and 
Acronyms from [Previous Figure] we can see that, barring the table values 
themselves, they are almost identical. The Table of Symbols is simply a table 
of values, essentially akin to a glossary, specific to the symbols that appear 
throughout the rest of the document. For each of those symbols, we see the 
symbol itself, a brief description of what that symbol represents, and the units
it is measured in (if applicable). Similarly, the Table of Units lists the
Système International d’Unités (SI) Units used throughout the document, their
descriptions, and the SI name. Finally, the table of Abbreviations and Acronyms
merely lists the abbreviations and their full forms, which are essentially the
symbols and their descriptions.

The reference section of the SRS provides a lot of knowledge, in a very 
straightforward and organized manner. The basic units provided in the table of 
units give a prime example of fundamental / global knowledge. Nearly any system
involving physical quantities will use some of these units. On the other hand, 
the table of symbols provides system/problem-domain specific knowledge that
will not be useful across unrelated domains. For example, the stress 
distribution factor (J) from GlassBR may appear in several related problems, but
would be unlikely to be seen in something like SWHS/NoPCM or Projectile. 
Finally, acronyms are very context-dependant. They are often specific to a
given domain and, without a coinciding definition, it can be very difficult for
even the target audience to understand what they refer to. Within one domain, 
there may be several acronyms meaning different things (for example: PM can 
refer to Product Manager, Project Manager, Program Manager, etc).

From the Reference section alone, 


- Practical approach to design
- Let's use some case studies
- Understanding what's really going on

\section{Artifact Summary}

See Table [NUM] for a summary of each software artifact's 'what', 'who', and
'how'.

*TABLE*
| Artifact | Who (Audience) | What (Problem) | How (Specific vs boilerplate) |

\section{Patterns and repetition and patterns and repetition -- (OR -- Repeating patterns and patterns that repeat --)}
From the above sections, we see many emerging patterns in our software
artifacts. Ignoring, for now, the organizational patterns from the [SmithEtAl]
templates we can already see simple patterns emerging. For example, we see the
same concept being introduced in multiple areas within a single artifact and
across artifacts in a project. [Example from one of the figures in the previous
section. Preferably something like a DD or TM that shows up within a single doc
multiple times]. We also see patterns of commonality across software family
members (The SWHS and NoPCM case studies) as they have been developed to solve
similar, or in our case nearly identical, problems.

- inter-project (repetition throughout different views + other patterns.)
  vs intra-project knowledge (repetition across projects/family members,
  minor modifications, but fundamentally the same + other patterns.)
- Hint at chunkifying/parceling out the fundamental (system/view-agnostic)
knowledge vs the specific knowledge

\section{Organizing knowledge - a fluid approach}
  **Subsec roadmap:
    - We see the patterns above, we can generalize a lot of that
    - Direct repetition (copy-paste) vs indirect repetition (view-changes)
    require us to pull together knowledge from all artifacts into one place
    - Some can be derived automatically, the rest must be explicitly stated
    - We need to create a categorization system (hint at chunks) that is both
    robust and extensible to cover a wide variety of use cases.
    - Finally the templates give us structure

  **NOTE: Under the hood section should explain the process of how we determined
  what we needed to do. What we ended up doing should come in the following
  section(s) - no 'real' implementation details, only conceptual stuff here.
  
\section{The seeds of Drasil}
  **Subsec roadmap:
    -- Summarize the above subsections and lead into next section
    -- Add relevant information that doesn't quite fit above 
      and isn't implementation related
    -- 'Relevant buckshot section'